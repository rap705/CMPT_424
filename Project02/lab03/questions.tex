%%%%%%%%%%%%%%%%%%%%%%%%%%%%%%%%%%%%%%%%%
%
% CMPT 424
% Fall 2020
% Lab Three
%
%%%%%%%%%%%%%%%%%%%%%%%%%%%%%%%%%%%%%%%%%

%%%%%%%%%%%%%%%%%%%%%%%%%%%%%%%%%%%%%%%%%
% Short Sectioned Assignment
% LaTeX Template
% Version 1.0 (5/5/12)
%
% This template has been downloaded from: http://www.LaTeXTemplates.com
% Original author: % Frits Wenneker (http://www.howtotex.com)
% License: CC BY-NC-SA 3.0 (http://creativecommons.org/licenses/by-nc-sa/3.0/)
% Modified by Alan G. Labouseur  - alan@labouseur.com
%
%%%%%%%%%%%%%%%%%%%%%%%%%%%%%%%%%%%%%%%%%

%----------------------------------------------------------------------------------------
%	PACKAGES AND OTHER DOCUMENT CONFIGURATIONS
%----------------------------------------------------------------------------------------

\documentclass[letterpaper, 10pt,DIV=13]{scrartcl} 

\usepackage[T1]{fontenc} % Use 8-bit encoding that has 256 glyphs
\usepackage[english]{babel} % English language/hyphenation
\usepackage{amsmath,amsfonts,amsthm,xfrac} % Math packages
\usepackage{sectsty} % Allows customizing section commands
\usepackage{graphicx}
\usepackage[lined,linesnumbered,commentsnumbered]{algorithm2e}
\usepackage{listings}
\usepackage{parskip}
\usepackage{lastpage}

\allsectionsfont{\normalfont\scshape} % Make all section titles in default font and small caps.

\usepackage{fancyhdr} % Custom headers and footers
\pagestyle{fancyplain} % Makes all pages in the document conform to the custom headers and footers

\fancyhead{} % No page header - if you want one, create it in the same way as the footers below
\fancyfoot[L]{} % Empty left footer
\fancyfoot[C]{} % Empty center footer
\fancyfoot[R]{page \thepage\ of \pageref{LastPage}} % Page numbering for right footer

\renewcommand{\headrulewidth}{0pt} % Remove header underlines
\renewcommand{\footrulewidth}{0pt} % Remove footer underlines
\setlength{\headheight}{13.6pt} % Customize the height of the header

\numberwithin{equation}{section} % Number equations within sections (i.e. 1.1, 1.2, 2.1, 2.2 instead of 1, 2, 3, 4)
\numberwithin{figure}{section} % Number figures within sections (i.e. 1.1, 1.2, 2.1, 2.2 instead of 1, 2, 3, 4)
\numberwithin{table}{section} % Number tables within sections (i.e. 1.1, 1.2, 2.1, 2.2 instead of 1, 2, 3, 4)

\setlength\parindent{0pt} % Removes all indentation from paragraphs.

\binoppenalty=3000
\relpenalty=3000

%----------------------------------------------------------------------------------------
%	TITLE SECTION
%----------------------------------------------------------------------------------------

\newcommand{\horrule}[1]{\rule{\linewidth}{#1}} % Create horizontal rule command with 1 argument of height

\title{	
   \normalfont \normalsize 
   \textsc{CMPT xxx - Fall 2020 - Dr. Labouseur} \\[10pt] % Header stuff.
   \horrule{0.5pt} \\[0.25cm] 	% Top horizontal rule
   \huge Lab One  \\     	    % Assignment title
   \horrule{0.5pt} \\[0.25cm] 	% Bottom horizontal rule
}

\author{Ryan Popeil \\ \normalsize ryan.popeil1@Marist.edu}

\date{\normalsize\9/28} 	% Today's date.

\begin{document}
\maketitle % Print the title

%----------------------------------------------------------------------------------------
%   start PROBLEM ONE
%----------------------------------------------------------------------------------------
\section{Problem One}
Internal fragmentation occurs when a program needs more memory than it has available.  This normally occurs when 
memory is broken into different fragements but the thing that needs to be stored in that fragment is larger than the fragement.
This differs from external fragmentation in which there is enough space to run the program however it is not continugious in memory.
When this happens the program is segmented throughout memory to make it easier to be stored.

%----------------------------------------------------------------------------------------
%   end PROBLEM ONE
%----------------------------------------------------------------------------------------

\pagebreak

%----------------------------------------------------------------------------------------
%   start PROBLEM TWO
%----------------------------------------------------------------------------------------
\section{Problem Two}
Optimal fit would go 300, 500, 200, 600.
First fit would go 600, 500, 300, and nothing would be large enough for the 426.
Best fit would go 300, 500, 200, 600.
Worst fit would go 600, 500, 300 and nothing would be left for the final 426.
%----------------------------------------------------------------------------------------
%   end PROBLEM Two
%----------------------------------------------------------------------------------------

\pagebreak


%----------------------------------------------------------------------------------------
%   REFERENCES
%----------------------------------------------------------------------------------------
% The following two commands are all you need in the initial runs of your .tex file to
% produce the bibliography for the citations in your paper.
\bibliographystyle{abbrv}
\bibliography{lab01} 
% You must have a proper ".bib" file and remember to run:
% latex bibtex latex latex
% to resolve all references.

\pagebreak

%----------------------------------------------------------------------------------------
%   Appendix
%----------------------------------------------------------------------------------------

\section{Appendix}

\subsection{Some JavaScript source code listings}

\lstset{numbers=left, numberstyle=\tiny, stepnumber=1, numbersep=5pt, basicstyle=\footnotesize\ttfamily}
\begin{lstlisting}[frame=single, ]  
var A = [5,0,5,6,6,8,45,50];

function solve(A) {
    // Base case to stop the recursion.
    if (A.length == 1) {
        if (A[0] % 5 == 0) {
            var retVal = 1;
        } else {
            var retVal = 0;
        }
        return retVal;
    } else {
        // Divide.
        var divPoint = Math.floor( A.length / 2);
        var left  = A.slice(0, divPoint);
        var right = A.slice(divPoint, A.length);
        
        // Conquer.
        var left5s   = solve(left, level+1);
        var center5s = straddle(left, right);
        var right5s  = solve(right, level+1);             
        
        // Combine.
        return Math.max(left5s, Math.max(center5s, right5s));
    }
}

function straddle(left, right) {
    var retVal = 0;
    if ((left[left.length-1] % 5 == 0) && (right[0] % 5 == 0)) {
        // Count back the 5's on the left going from right to left.
        var leftCount = 0;
        var index = left.length-1;
        while ( (index >= 0) && (left[index] % 5 == 0) ) {
            index--;
            leftCount++;
        }
        // Count forward the 5's on the right going from left to right.
        var rightCount = 0;
        while ( (rightCount < right.length) && (right[rightCount] % 5 == 0) ) {
            rightCount++;
        }
        // Return the sum of the straddling 5s on the left and right.
        retVal = leftCount + rightCount;
    }
    return retVal;
}
\end{lstlisting}

\end{document}
