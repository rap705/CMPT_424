\documentclass{article}

\begin{document}
1. There are number of advantages when using a same system call interface to manipulate files and devices.  The main advantage of this interface is that 
the devices are seen as files and are therefore handled by the kernel.  This makes the code much more portable across different OS because the user can find
an API so that thier device can be handled rather than it having to be written hardware specific into the OS.  Most modern OS use same system call interfaces
for this exact reason.  Although this is a good thing it does lead to one major disadvantage which is that it can become much harder or impossible to support
all of the functionality of the device although many OS have found ways to overcome this or limit its impact.

2.It would be possible for a user to create a new command interpreter using the system call interface.  The user could do this because they are given the ability
to create new processes and manage ones that are currently happening.  With all of this power it would be possible for a user to create a new command interpreter.

\end{document}